\begin{frame}{Alphabets and strings}
  In some contexts:
  \begin{itemize}
    \item Sets are called alphabets.
    \item Tuples over alphabets are called strings or words.
    \item We omit the brackets and commas when we can.
    \item The empty tuple is called the empty string and denoted $\epsilon$.
  \end{itemize}
  \vspace{3mm}
  \begin{exampleblock}{Example}
    Let $A$ be the alphabet $\{0, 1\}$. Each of the following is a string over that alphabet:
      $$ \epsilon, 0, 1, 00, 01, 10, 11, 001, 010, 011, 100, \ldots $$
  \end{exampleblock}
\end{frame}

\begin{frame}{Concatenation of strings}
  We can take two strings over the same alphabet and concatenate them.
  \vspace{2mm}
  \begin{exampleblock}{Example}
    \begin{itemize}
      \setlength\itemsep{3mm}
      \item Let $s_1 = 00$ and $s_2=101$ be two strings over $\{0,1\}$.
      \item Concatenating $00$ and $101$ gives $s_1 \circ s_2 = 00101$.
      \item Technically, $(0,0) \circ (1,0,1) = (0,0,1,0,1)$.
      \item When we can, we omit the notation: $s_1s_2 = 00101$.
      \item Also, $s_2s_1 = 10100$ which is a different string.
      \item Concatenation is not commutative.
    \end{itemize}
  \end{exampleblock}
\end{frame}


\begin{frame}{Kleene star of an alphabet}
  \redmath{A = \{0,1\}}
  
  \begin{description}[abc]
    \item[$A^1$] $= \{0,1\}$, the strings of length one over $A$.
    \item[$A^2$] $= \{00,01,10,11\}$, the strings of length two over $A$.
    \item[$A^i$] $=$ the strings of length $i$ over $A$, $i \in \mathbb{N}_0$.
    \item[$A^0$] $= \{\epsilon\}$, the strings of length zero over $A$.
  \end{description}

  \begin{exampleblock}{Definition}
    The Kleene star of $A$ is the union of all the $A^i$:
    \redmath{A^* = \bigcup_i A^i = \{ \epsilon, 0, 1, 00, 01, 10, \ldots\}}
  \end{exampleblock}

  \emph{Note this applies to any alphabet, not just $\{0,1\}$.}
\end{frame}


\begin{frame}{Languages}
  \redmath{L \subseteq A^*}

  A subset of the star of an alphabet is a \textbf{language} over it.

  \begin{exampleblock}{Example}
    \begin{itemize}
      \item Let $A = \{a,b,c\}$.
      \item Then $A^*=\{\epsilon,a,b,c,aa,ab,ac,ba,bb,bc,\ldots\}$.
      \item $L_1 = \{aa,bbb,ccc\}$ is a language.
      \item $L_2 = \{s \mid s \textrm{ contains an } a \}$ is also a language.
      \item Read this as ``all strings $s$ where $s$ contains an $a$''.
    \end{itemize}
  \end{exampleblock}

  We can create new languages from smaller ones using union, concatenate and star.
\end{frame}


\begin{frame}{Union of languages}
  \redmath{L_1 = \{00,11\} \quad L_2 = \{11,111,1111\}}
  \redmath{L_1 \cup L_2 = \{00,11,111,1111\}}
  
  \begin{itemize}
    \setlength\itemsep{4mm}
    \item The union of languages is the set of all strings in any of them.
    \item We could consider two languages over different alphabets, if we took the union of the alphabets -- we usually don't.
  \end{itemize}
\end{frame}


\begin{frame}{Concatenation of languages}
  \redmath{L_1 = \{00,11\} \quad L_2 = \{11,111,1111\}}
  \redmath{L_1 \circ L_2 = \{0011,00111,00111111,1111,11111,11111111\}}
  \begin{itemize}
    \setlength\itemsep{4mm}
    \item The concatenation of two languages is the set of concatenations of each string from the first language with each string from the second language.
    \item Note, usually $L_1 \circ L_2 \neq L_2 \circ L_1$. When are they equal?
    \item We usually omit the circle: $L_1 L_2$.
  \end{itemize}
\end{frame}


\begin{frame}{Star of languages}
  \redmath{L = \{00,11\}}
  \redmath{L^* = \{\epsilon,00,11,0000,0011,1100,1111,\ldots\}}
  \begin{itemize}
    \item The star of a language is the same idea as the star of an alphabet.
    \item $L^1$ is the language itself.
    \item $L^2$ is the language concatenated to itself.
    \item $L^3$ is the language concatenated to itself twice.
    \item $L^0$ is the language containing only $\epsilon$.
    \item The star is the union of $L^i$ for all $i \in \mathbb{N}_0$.
    \item We also define $L^+$ as the union of $L^i$ for all $i \in \mathbb{N}$.
  \end{itemize}
\end{frame}


\begin{frame}{File types as languages}
  \begin{itemize}
    \setlength\itemsep{3mm}
    \item The set of all valid pdf files is a language over the alphabet $A =\{0,1\}$.
    \item So, the set of valid pdf's is a subset of $A^*$.
    \item As is the set of valid docx files.
    \item A computer program that converts pdf's to docx's maps one subset of $A^*$ to another.
    \item Remember too, that executable files themselves are stored as files in 0's and 1's.
    \item In fact, all files are in $\{0,1\}^*$.
  \end{itemize}
\end{frame}